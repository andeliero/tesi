%**************************************************************
% file contenente le impostazioni della tesi
%**************************************************************

%**************************************************************
% Frontespizio
%**************************************************************
\newcommand{\myName}{Alberto Andeliero }                         % autore
\newcommand{\myTitle}{Tree recommendation system}                    
\newcommand{\myDegree}{Tesi di laurea triennale}                % tipo di tesi
\newcommand{\myUni}{Università degli Studi di Padova}           % università
\newcommand{\myFaculty}{Corso di Laurea in Informatica}         % facoltà
\newcommand{\myDepartment}{Dipartimento di Matematica}          % dipartimento
\newcommand{\myProf}{Tullio Vardanega }                          % relatore
\newcommand{\myLocation}{Padova}                                % dove
\newcommand{\myAA}{2015-2016}                                   % anno accademico
\newcommand{\myTime}{Jan 2016}                                  % quando


%**************************************************************
% Impostazioni di impaginazione
% see: http://wwwcdf.pd.infn.it/AppuntiLinux/a2547.htm
%**************************************************************

\setlength{\parindent}{14pt}   % larghezza rientro della prima riga
\setlength{\parskip}{0pt}   % distanza tra i paragrafi


%**************************************************************
% Impostazioni di biblatex
%**************************************************************
\bibliography{bibliografia} % database di biblatex 

\defbibheading{bibliography}
{
    \cleardoublepage
    \phantomsection 
    \addcontentsline{toc}{chapter}{\bibname}
    \chapter*{\bibname\markboth{\bibname}{\bibname}}
}

\setlength\bibitemsep{1.5\itemsep} % spazio tra entry

\DeclareBibliographyCategory{opere}
\DeclareBibliographyCategory{web}

\addtocategory{opere}{womak:lean-thinking}
\addtocategory{web}{site:agile-manifesto}

\defbibheading{opere}{\section*{Riferimenti bibliografici}}
\defbibheading{web}{\section*{Siti Web consultati}}


%**************************************************************
% Impostazioni di caption
%**************************************************************
\captionsetup{
    tableposition=top,
    figureposition=bottom,
    font=small,
    format=hang,
    labelfont=bf
}

%**************************************************************
% Impostazioni di glossaries
%**************************************************************

%**************************************************************
% Acronimi
%**************************************************************
\renewcommand{\acronymname}{Acronimi e abbreviazioni}

\newacronym[description={\glslink{umlg}{Unified Modeling Language}}]
    {uml}{UML}{Unified Modeling Language}

\newacronym[description={\glslink{ictg}{Information Communication Technology}}]
		{ict}{ICT}{Information Communication Technology}

\newacronym[description={\glslink{cmsg}{Content Management System}}]
		{cms}{CMS}{Content Management System}

\newacronym[description={\glslink{mvcg}{Model View Controller}}]
    {mvc}{MVC}{Model View Controller}

\newacronym[description={\glslink{daog}{Data Access Object}}]
    {dao}{DAO}{Data Access Object}

\newacronym[description={\glslink{ideg}{Integrated Development Environment}}]
    {ide}{IDE}{Integrated Development Environment}


\newacronym[description={\glslink{apig}{Application Program Interface}}]
    {api}{API}{Application Program Interface}

\newacronym[description={\glslink{jsong}{Javascript Object Notation}}]
    {json}{JSON}{Javascript Object Notation}

\newacronym[description={\glslink{restg}{Representational State Transfer}}]
    {rest}{REST}{Representational State Transfer}


%**************************************************************
% Glossario
%**************************************************************
\renewcommand{\glossaryname}{Glossario}

\newglossaryentry{core business}
{
    name=\glslink{core business}{Core business},
    text=core business,
    sort=core business,
    description={Il core business è l'attività principale di un'azienda, ovvero la base su cui si basa il fatturato e il guadagno a fine esercizio}
}

\newglossaryentry{ictg}
{
		name=\glslink{ict}{ICT},
		text=Information Communication Technology,
		sort=ict,
		description={sono l'insieme dei metodi e delle tecnologie che realizzano i sistemi di trasmissione, ricezione ed elaborazione di informazioni (tecnologie digitali comprese)}
}

\newglossaryentry{cmsg}
{
    name=\glslink{cms}{CMS},
    text=Content Management System,
    sort=content management system,
    description={strumento software installato su un server web studiato per facilitare la
gestione dei contenuti dei siti web, svincolando l’amministratore da conoscenze
tecniche di programmazione}
}

\newglossaryentry{mvcg}
{
    name=\glslink{mvc}{MVC},
    text=Model View Controller,
    sort=model view controller,
    description={il Model-View-Controller, in informatica, è un pattern architetturale molto diffuso nello sviluppo di sistemi software, in particolare nell’ambito della programmazione orientata agli oggetti, in grado di separare la logica di presentazione dei dati dalla logica di business}
}

\newglossaryentry{framework}
{
    name=\glslink{framework}{Framework},
    text=framework,
    sort=framework,
    description={in informatica, e specificatamente nello sviluppo software, un framework è un’architettura logica di supporto su cui un software può essere progettato e realizzato, spesso facilitandone lo sviluppo da parte del programmatore}
}

\newglossaryentry{daog}
{
    name=\glslink{dao}{DAO},
    text=Data Access Object,
    sort=data access object,
    description={è un pattern architetturale per la gestione della persistenza: si tratta fondamentalmente di una classe con relativi metodi che rappresenta un’entità tabellare di un RDBMS, usata principalmente in applicazioni web, per stratificare e isolare l’accesso ad una tabella/record tramite query (poste all’interno dei metodi della classe) ovvero al data layer da parte della business logic creando un maggiore livello di astrazione ed una più facile manutenibilità. I metodi del DAO con le rispettive query dentro verranno così richiamati dalle classi della business logic}
}

\newglossaryentry{ideg}
{
    name=\glslink{ide}{IDE},
    text=Integrated Development Environment,
    sort=Integrated Development Environment,
    description={software che fornisce allo sviluppatore numerosi aiuti durante lo sviluppo del codice. Esempi di aiuti sono la segnalazione degli errori di sintassi, interazione con strumenti di debug, suggerimenti di completamento del codice e strumenti per le build automatiche}
}

\newglossaryentry{apig}
{
    name=\glslink{api}{API},
    text=Application Program Interface,
    sort=api,
    description={in informatica con il termine \emph{Application Programming Interface API} (ing. interfaccia di programmazione di un'applicazione) si indica ogni insieme di procedure disponibili al programmatore, di solito raggruppate a formare un set di strumenti specifici per l'espletamento di un determinato compito all'interno di un certo programma. La finalità è ottenere un'astrazione, di solito tra l'hardware e il programmatore o tra software a basso e quello ad alto livello semplificando così il lavoro di programmazione}
}

\newglossaryentry{umlg}
{
    name=\glslink{uml}{UML},
    text=Unified Modeling Language,
    sort=uml,
    description={in ingegneria del software UML, Unified Modeling Language è un linguaggio di modellazione e specifica basato sul paradigma object-oriented. L'UML svolge un’importantissima funzione di "lingua franca" nella comunità della progettazione e programmazione a oggetti. Gran parte della letteratura di settore usa tale linguaggio per descrivere soluzioni analitiche e progettuali in modo sintetico e comprensibile a un vasto pubblico}
}

\newglossaryentry{jsong}
{
    name=\glslink{json}{JSON},
    text=Javascript Object Notation,
    sort=json,
    description={formato adatto per lo scambio dei dati in applicazioni client-server. Presenta una struttura semplice e di facile comprensione per l’uomo. Tale caratteristica ha contribuito alla sua rapida diffusione.}
}

\newglossaryentry{restg}
{
    name=\glslink{rest}{REST},
    text=Representational State Transfer,
    sort=rest,
    description={un tipo di architettura software per i sistemi di ipertesto distribuiti come il World Wide Web}
}
 % database di termini
\makeglossaries


%**************************************************************
% Impostazioni di graphicx
%**************************************************************
\graphicspath{{immagini/}} % cartella dove sono riposte le immagini


%**************************************************************
% Impostazioni di hyperref
%**************************************************************
\hypersetup{
    %hyperfootnotes=false,
    %pdfpagelabels,
    %draft,	% = elimina tutti i link (utile per stampe in bianco e nero)
    colorlinks=true,
    linktocpage=true,
    pdfstartpage=1,
    pdfstartview=FitV,
    % decommenta la riga seguente per avere link in nero (per esempio per la stampa in bianco e nero)
    %colorlinks=false, linktocpage=false, pdfborder={0 0 0}, pdfstartpage=1, pdfstartview=FitV,
    breaklinks=true,
    pdfpagemode=UseNone,
    pageanchor=true,
    pdfpagemode=UseOutlines,
    plainpages=false,
    bookmarksnumbered,
    bookmarksopen=true,
    bookmarksopenlevel=1,
    hypertexnames=true,
    pdfhighlight=/O,
    %nesting=true,
    %frenchlinks,
    urlcolor=webbrown,
    linkcolor=RoyalBlue,
    citecolor=webgreen,
    %pagecolor=RoyalBlue,
    %urlcolor=Black, linkcolor=Black, citecolor=Black, %pagecolor=Black,
    pdftitle={\myTitle},
    pdfauthor={\textcopyright\ \myName, \myUni, \myFaculty},
    pdfsubject={},
    pdfkeywords={},
    pdfcreator={pdfLaTeX},
    pdfproducer={LaTeX}
}

%**************************************************************
% Impostazioni di itemize
%**************************************************************
\renewcommand{\labelitemi}{$\ast$}

%\renewcommand{\labelitemi}{$\bullet$}
%\renewcommand{\labelitemii}{$\cdot$}
%\renewcommand{\labelitemiii}{$\diamond$}
%\renewcommand{\labelitemiv}{$\ast$}


%**************************************************************
% Impostazioni di listings
%**************************************************************
\lstset{
    language=[LaTeX]Tex,%C++,
    keywordstyle=\color{RoyalBlue}, %\bfseries,
    basicstyle=\small\ttfamily,
    %identifierstyle=\color{NavyBlue},
    commentstyle=\color{Green}\ttfamily,
    stringstyle=\rmfamily,
    numbers=none, %left,%
    numberstyle=\scriptsize, %\tiny
    stepnumber=5,
    numbersep=8pt,
    showstringspaces=false,
    breaklines=true,
    frameround=ftff,
    frame=single
} 


%**************************************************************
% Impostazioni di xcolor
%**************************************************************
\definecolor{webgreen}{rgb}{0,.5,0}
\definecolor{webbrown}{rgb}{.6,0,0}


%**************************************************************
% Altro
%**************************************************************

\newcommand{\omissis}{[\dots\negthinspace]} % produce [...]

% eccezioni all'algoritmo di sillabazione
\hyphenation
{
    ma-cro-istru-zio-ne
    gi-ral-din
}

\newcommand{\sectionname}{sezione}
\addto\captionsitalian{\renewcommand{\figurename}{figura}
                       \renewcommand{\tablename}{tabella}}

\newcommand{\glsfirstoccur}{\ap{{[g]}}}

\newcommand{\intro}[1]{\emph{\textsf{#1}}}

%**************************************************************
% Environment per ``rischi''
%**************************************************************
\newcounter{riskcounter}                % define a counter
\setcounter{riskcounter}{0}             % set the counter to some initial value

%%%% Parameters
% #1: Title
\newenvironment{risk}[1]{
    \refstepcounter{riskcounter}        % increment counter
    \par \noindent                      % start new paragraph
    \textbf{\arabic{riskcounter}. #1}   % display the title before the 
                                        % content of the environment is displayed 
}{
    \par\medskip
}

\newcommand{\riskname}{Rischio}

\newcommand{\riskdescription}[1]{\textbf{\\Descrizione:} #1.}

\newcommand{\risksolution}[1]{\textbf{\\Soluzione:} #1.}

%**************************************************************
% Environment per ``use case''
%**************************************************************
\newcounter{usecasecounter}             % define a counter
\setcounter{usecasecounter}{0}          % set the counter to some initial value

%%%% Parameters
% #1: ID
% #2: Nome
\newenvironment{usecase}[2]{
    \renewcommand{\theusecasecounter}{\usecasename #1}  % this is where the display of 
                                                        % the counter is overwritten/modified
    \refstepcounter{usecasecounter}             % increment counter
    \vspace{10pt}
    \par \noindent                              % start new paragraph
    {\large \textbf{\usecasename #1: #2}}       % display the title before the 
                                                % content of the environment is displayed 
    \medskip
}{
    \medskip
}

\newcommand{\usecasename}{UC}

\newcommand{\usecaseactors}[1]{\textbf{\\Attori Principali:} #1. \vspace{4pt}}
\newcommand{\usecasepre}[1]{\textbf{\\Precondizioni:} #1. \vspace{4pt}}
\newcommand{\usecasedesc}[1]{\textbf{\\Descrizione:} #1. \vspace{4pt}}
\newcommand{\usecasepost}[1]{\textbf{\\Postcondizioni:} #1. \vspace{4pt}}
\newcommand{\usecasealt}[1]{\textbf{\\Scenario Alternativo:} #1. \vspace{4pt}}

%**************************************************************
% Environment per ``namespace description''
%**************************************************************

\newenvironment{namespacedesc}{
    \vspace{10pt}
    \par \noindent                              % start new paragraph
    \begin{description} 
}{
    \end{description}
    \medskip
}

\newcommand{\classdesc}[2]{\item[\textbf{#1:}] #2}